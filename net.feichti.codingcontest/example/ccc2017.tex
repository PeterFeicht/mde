\documentclass[a4paper,11pt]{article}

\usepackage[american]{babel}
\usepackage[T1]{fontenc}
\usepackage[utf8]{inputenc}
\usepackage[babel]{csquotes}
\usepackage{lmodern}

% Some geometry customization
\usepackage[margin=2cm]{geometry}
\setlength{\parindent}{0pt}
\setlength{\parskip}{0.5em plus 1.5pt minus 1.5pt}
\setlength{\headheight}{15pt}
\setlength{\tabcolsep}{8pt}

% Make enumerations a little more compact
\usepackage{enumitem}
\setlist{topsep=0pt, itemsep=2pt}
\setlist[1]{labelindent=6pt}

\usepackage[
  pdflang=en,
  colorlinks=true,
  unicode=true,
  pdfstartview=,
  linkcolor=black,
  filecolor=black,
  menucolor=black,
  urlcolor=black
]{hyperref}
\hypersetup{
  pdftitle={Coding Contest: Hyperloop - Global Metro},
  pdfauthor={Organized by Catalysts}
}

\usepackage{fancyhdr}
\pagestyle{fancy}


\begin{document}

\title{Hyperloop - Global Metro}
\author{Organized by \href{https://contest.catalysts.cc/}{Catalysts}}
\date{Local date: Fri Mar 31 13:00:00 CEST 2017}
\maketitle
\vspace{1em}


\section{Locations}

This is a list of all physical locations the contest will be held at. In addition to that, online entries will be
accepted also.

\begin{description}
  \item[Cluj, RO] Babes-Bolyai-University, Str. Teodor Mihali NR 58-60, Cluj-Napoca
  \item[Den Haag, NL] Leiden University Campus The Hague, Turfmarkt 99, 2511 DP The Hague
  \item[Johannesburg, ZA] Honours Laboratory E-Ring 212, APK Campus University of Johannesburg, Cnr. Kingsway and University Road Auckland Park
  \item[Linz, AT] Johannes Kepler Universität, Altenberger Straße 69, 4040 Linz
  \item[Wien, AT] Stadthalle, Friedrich-Schmidt-Platz 1, 1010 Wien
\end{description}


\section{Levels}

The following sections will describe each level in detail.

\subsection{Level 1 - Connection Time}

\begin{description} % TODO Parse lists in the description to generate LaTeX lists
  \item[en] \emph{Connection Time} --- 
    Your task in this level is to estimate the travel time for a direct hyperloop
    connection.
    
    The input consists of a file containing locations and a direct hyperloop
    connection:
    
     * A location is a named (x, y) position on a map. The coordinates are in
       meters.
     * The hyperloop connection directly connects exactly two locations. There are
       no intermediate stops.
    
    Some locations may be unused; just ignore them.
    The distance between two locations is the ordinary (Cartesian) distance. That
    is, the world is flat and there are no obstacles.
    In our model, the hyperloop travels at a constant speed of 250 m/s and waits for
    200 seconds at each stop. The hyperloop travel time includes the wait time at
    the start location of the journey. The trip is over at the end location - no
    more waiting time there.
    You should output the hyperloop connection travel time (in seconds), rounded to
    the nearest integer.
  
  \item[de] \emph{Verbindungszeit} --- 
    Deine Aufgabe in diesem Level ist, die Reisezeit für eine direkte Hyperloop
    Verbindung zu schätzen.
    
    Die Eingabe besteht aus einer Datei mit Orten von Haltestellen und einer
    direkten Verbindung:
    
     * Eine Haltestelle ist eine Position (x, y) mit Namen auf einer Karte. Die
       Koordinaten sind in Meter.
     * Die Hyperloop-Verbindung verbindet zwei Haltestellen ohne Zwischenstop.
    
    Einige Haltestellen sind möglicherweise unbenutzt, diese können ignoriert
    werden.
    Die Strecke zwischen zwei Haltestellen ist die einfache Karthesische Entfernung.
    Die Welt ist also flach und es gibt keine Hindernisse.
    In unserem Modell fährt der Hyperloop mit konstanten 250 m/s und bleibt an jeder
    Haltestelle 200 Sekunden stehen. Die Reisezeit enthält die Wartezeit an der
    Starthaltestelle, die Reise ist an der Endhaltestelle vorbei, dort gibt es keine
    Wartezeit mehr.
    Es soll die Reisezeit für die Verbindung in Sekunden ausgegeben werden, gerundet
    auf die nächste ganze Sekunde.
  
\end{description}

\subsection{Level 2 - Journey Time}

\begin{description} % TODO Parse lists in the description to generate LaTeX lists
  \item[en] \emph{Journey Time} --- 
    Your task in this level is to estimate the total duration of a journey which
    uses a direct hyperloop connection.
    The input now also includes a journey for a traveller wanting to get from a
    start location to an end location.
    You should output the duration of the journey (in seconds), rounded to the
    nearest integer.
    
    A journey using the hyperloop is made up of 3 parts:
    
     1. Driving from the journey start location to the closest stop of the two
        locations that make up the hyperloop connection
     2. Travelling with the hyperloop in whichever direction is necessary
     3. Driving from the other stop of the hyperloop connection to the journey end
        location
    
    In our model, travellers drive at a constant 15 m/s. They are always able to
    drive directly in a straight line to and from hyperloop locations.
    It will never be faster to drive directly from the start to the end location
    than to use the hyperloop.
  
  \item[de] \emph{Reisezeit} --- 
    Deine Aufgabe in disem Level ist, die gesamte Reisezeit zu schätzen, wenn eine
    direkte Verbindung mit dem Hyperloop benutzt wird.
    Die Eingabe enthält jetzt auch eine gewünschte Reisestrecke vor der Hyperloop-
    Verbindung.
    Es soll die gesamte Reisezeit in Sekunden ausgegeben werden, gerundet auf die
    nächste ganze Sekunde.
    
    Eine Reise mit dem Hyperloop besteht nun aus drei Teilen:
    
     1. Fahrt vom Startort zur nächsten Hyperloop Haltestelle
     2. Fahr mit dem Hyperloop zur anderen Haltestelle
     3. Fahrt von der Endhaltestelle zum Reiseziel
    
    In unserem Modell fahren die Reisenden mit konstanten 15 m/s, immer in einer
    geraden Linie von einem Ort zu einer Haltestelle. Es ist nie schneller, direkt
    ohne Hyperloop vom Start- zum Zielort zu fahren.
  
\end{description}

\subsection{Level 3 - Connection Evaluation}

\begin{description} % TODO Parse lists in the description to generate LaTeX lists
  \item[en] \emph{Connection Evaluation} --- 
    A hyperloop connection has been proposed. Your task is to help work out whether
    this connection is worthwhile.
    The input is similar to Level 2, except that many journeys are provided. In
    addition, the time in seconds required to complete each journey using existing
    transport options (the current time) is provided. The current time is always
    shorter than the direct driving time, using our driving model.
    Drivers are expected to switch to using the hyperloop line if it makes their
    journey faster.
    You should output the number of journeys for which the hyperloop journey is
    faster than the current time.
  
  \item[de] \emph{Verbindung Bewerten} --- 
    Es ist nun eine bestimmte Hyperloop-Verbindung vorgeschlagen worden. Deine
    Aufgabe ist es festzustellen, ob sich diese Verbindung rentieren würde.
    Die Eingabe ist ähnlich zum Level 2, mit dem Unterschied, dass jetzt eine
    Vielzahl von Reisestrecken vorgegeben ist. Zusätzlich ist die Zeit abgegeben,
    die ohne Hyperloop mit vorhandenen Transportmitteln für eine Reise benötigt
    wird. Die aktuelle Zeit ist immer kürzer als eine direkte Fahrverbindung.
    Fahrer werden natürlich den Hyperloop verwenden, falls das die Reisezeit
    verkürzt.
    Es soll jetzt die Anzahl der Reisestrecken ausgegeben werden, für die die
    Reisezeit mit dem Hyperloop kürzer wird.
  
\end{description}

\subsection{Level 4 - Connection Proposal}

\begin{description} % TODO Parse lists in the description to generate LaTeX lists
  \item[en] \emph{Connection Proposal} --- 
    Now it is your turn to propose a direct hyperloop connection.
    The input is similar to Level 3, but you won’t be given a hyperloop connection.
    Instead you will be given a target number, N, of journeys to benefit from the
    hyperloop.
    You should output a hyperloop connection. Of the input journeys, at least N
    must be faster using your hyperloop connection than currently. The hyperloop
    journey time is given by the rules from Levels 1 and 2. There may be multiple
    valid solutions, but you only need to find one.
  
  \item[de] \emph{Verbindung Vorschlagen} --- 
    Nun ist es an dir, eine neue Hyperloop-Verbindung vorzuschlagen.
    Die Eingabe ist ähnlich wie im Level 3, aber es ist keine Hyperloop-Verbindung
    vorgegeben. Stattdessen ist eine Mindestanzahl N von Reisestrecken vorgegeben,
    die mit der neuen Hyperloop-Verbindung schneller sein sollen als vorher.
    Es soll jetzt die Hyperloop-Verbindung ausgegeben werden. Mit dieser Verbindung
    müssen mindestens N Reisestrecken in kürzerer Zeit zu bewältigen sein. Es gibt
    möglicherweise mehrere mögliche Verbindungen, es muss nur eine gefunden werden.
  
\end{description}


\section{Teams}

The following section lists all registered teams and their participants, grouped by their entry location.

\subsection{Online Entries}

\begin{description}
  \item[Alessio]
    Single entry (30 years old)
  \item[Markus Zancolo]
    Single entry (23 years old)
  \item[DataVenture]
    Team:
    \begin{itemize}
      \item \textbf{jjerphan} 
      \item \textbf{Mathis Chenuet} 
    \end{itemize}
\end{description}

\subsection{Linz}

\begin{description}
  \item[Trust us! We are engineers!]
    Team:
    \begin{itemize}
      \item \textbf{Simon Lehner-Dittenberger} (27 years old)
      \item \textbf{AwesomeDragon} (26 years old)
    \end{itemize}
  \item[Christian Bartsch]
    Single entry (22 years old)
\end{description}

\subsection{Wien}

\begin{description}
  \item[die\_drei\_beiden]
    Team:
    \begin{itemize}
      \item \textbf{anonym1} (18 years old)
      \item \textbf{anonym2} (19 years old)
      \item \textbf{anonym3} (20 years old)
    \end{itemize}
\end{description}

\subsection{Johannesburg}

\begin{description}
  \item[mob\_justice]
    Team:
    \begin{itemize}
      \item \textbf{dube} (18 years old)
      \item \textbf{Thaps} (21 years old)
    \end{itemize}
\end{description}

\subsection{Cluj}

\begin{description}
  \item[UBB\_Random]
    Team:
    \begin{itemize}
      \item \textbf{Mihai Zsisku} (22 years old)
      \item \textbf{Muntea Andrei-Marius} (21 years old)
      \item \textbf{Sergiu-Catalin Maries} (22 years old)
    \end{itemize}
  \item[FastAndFourier]
    Team:
    \begin{itemize}
      \item \textbf{Emanuel Truta} (19 years old)
      \item \textbf{Lup Vasile} (21 years old)
    \end{itemize}
\end{description}

\subsection{Den Haag}

\begin{description}
  \item[We're having a field day]
    Team:
    \begin{itemize}
      \item \textbf{Daan van Gent} (24 years old)
    \end{itemize}
\end{description}


\end{document}
